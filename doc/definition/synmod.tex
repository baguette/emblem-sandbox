\section{Syntax of Modules}
\label{syn-mod-sec}
For Modules there are further reserved words, identifier classes and derived
forms. There are no further special constants; 
comments and lexical analysis are as for the Core.
\replacement{\thenostrsharing}{The derived forms for modules concern functors and  appear in 
Appendix~\ref{derived-forms-app}.}{The derived forms for modules  appear in 
Appendix~\ref{derived-forms-app}.}

%For Modules there are further keywords and identifier classes, but no
%further special constants and at present no further derived forms.  
%Comments and lexical analysis are as for the Core.

\subsection{Reserved Words}
The\index{14.1} following are the additional reserved words used in Modules.\note{\thenostrsharing}{Inserted new keywords  {\tt where} and {\tt :>}}
% \begin{verbatim}
%              eqtype  functor    include   sharing
%              sig     signature  struct  structure
% \end{verbatim}
\begin{verbatim}
             eqtype    functor   include   sharing   sig
             signature   struct   structure   where   :>
\end{verbatim}
\subsection{Identifiers}
\label{syn-mod-identifiers-sec}
The additional \replacement{\thefixtypos}{syntax}{identifier} classes for Modules are SigId (signature identifiers)
and FunId (functor identifiers)\replacement{\theidstatus}{; they may be either alphanumeric -- not
starting with a prime -- or symbolic}{. Functor and signature identifiers
must be alphanumeric, not starting with a prime.}  The class of each identifier occurrence
is determined by the grammatical rules which follow.  
Henceforth, therefore,
we consider all identifier classes to be disjoint.

\subsection{Infixed operators}
\label{infixopmod.sec}
In addition to the scope rules for fixity directives given for the Core syntax,
there is a further scope limitation:
if $\dir$ occurs in a structure-level declaration $\strdec$ in any of the 
phrases
\[ \LET\ \strdec\ \IN\ \cdots\ \END \]
\[ \LOCAL\ \strdec\ \IN\ \cdots\ \END \]
\[ \STRUCT\ \strdec\ \END \]
then the scope of $\dir$ does not extend beyond the phrase.

One effect of this limitation is that fixity is local to a \replacement{\thenostrsharing}{generative}{basic}
structure expression --- in particular, to such an expression occurring
as a functor body. 
\deletion{\thenoinfixinspec}{A more liberal scheme (which is under consideration)
would allow fixity directives to appear also as specifications, so that
fixity may be dictated by a signature expression; furthermore, it would allow an ~$\OPEN$~
or ~$\INCLUDE$~ construction to restore the fixity which prevailed
in the structures being opened, or in the signatures being included.
This scheme is not adopted at present.}


\subsection{Grammar for Modules}
\label{mod-gram-sec}
The\index{14.2} phrase classes for Modules are shown in Figure~\ref{mod-phr}.
We use the variable $\strexp$ to range over StrExp, etc.
The conventions adopted in presenting the grammatical 
%\pagebreak
\begin{figure}[t]
\vspace{4pt}
\makeatletter{}
\tabskip\@centering
\halign to\textwidth
{#\hfil\tabskip1em&#\hfil\tabskip\@centering\cr
StrExp & structure expressions \cr
StrDec & structure-level declarations \cr
StrBind & structure bindings \cr
\cr
SigExp & signature expressions \cr
SigDec & signature declarations \cr
SigBind & signature bindings \cr
\cr
Spec & specifications \cr
ValDesc & value descriptions\cr
TypDesc & type descriptions\cr
DatDesc & datatype descriptions\cr
ConDesc & constructor descriptions\cr
ExDesc & exception descriptions\cr
StrDesc & structure descriptions\cr
%\ adhocinsertion{\thetypabbr}{2mm}{WhereBind & where bindings\cr}
\adhocdeletion{\thetypabbr}{2mm}{SharEq & sharing equations\cr}
\cr
FunDec & functor declarations\cr
FunBind & functor bindings\cr\adhocdeletion{\thenofuncspec}{3mm}{FunSigExp & functor signature expressions\cr
FunSpec & functor specifications\cr
FunDesc & functor descriptions\cr}TopDec  & top-level declarations\cr
}
\makeatother
\caption{Modules Phrase Classes\index{15.1}}
\label{mod-phr}
\end{figure}
rules for Modules
are the same as for the Core.
The grammatical rules are shown in Figures~\ref{str-syn},
\ref{spec-syn} and \ref{prog-syn}.

\deletion{\thenofuncspec}{It should be noted that functor specifications (FunSpec) cannot
occur in programs;
neither can the associated functor descriptions (FunDesc)
and functor signature expressions (FunSigExp).  The purpose of a $\funspec$
is to specify the static attributes (i.e. functor signature) of one
or more functors. This will be useful, in fact essential, for
separate compilation of functors. If, for example, a functor $g$
refers to another functor $f$ then --- in order to compile $g$ in
the absence of the declaration of $f$ --- at least the specification
of $f$ (i.e. its functor signature) must be available. At present there is no
special grammatical form for  a separately compilable ``chunk'' of text
-- which we may like to call call a {\sl module} -- containing a $\fundec$
together with a $\funspec$ specifying its global references. However, below in
the semantics for Modules it is defined when a
declared functor matches a functor signature specified for it. This determines
exactly those functor environments (containing declared functors
such as $f$) into which the separately compiled ``chunk''
containing the declaration of $g$ may be loaded.}
%\newpage
\begin{figure}[h]
\vspace{4pt}
\makeatletter{}
\tabskip\@centering
\halign to\textwidth
{#\hfil\tabskip1em&\hfil$#$\hfil&#\hfil&#\hfil\tabskip\@centering\cr
\strexp & ::=   & \encstrexp    & \adhocreplacementl{\thenostrsharing}{-4cm}{generative}{basic}\cr
        &       & \longstrid    & structure identifier\cr
\adhocinsertion{\thenostrsharing}{0mm}{ && \transpconstraint & transparent constraint\cr}
\adhocinsertion{\thenostrsharing}{0mm}{ && \opaqueconstraint & opaque constraint\cr}
        &       & \funappstr    & functor application\cr
        &       & \letstrexp    & local declaration\cr
\noalign{\vspace{6pt}}
%
\strdec & ::=   & \dec                          & declaration \cr
        &       & \singstrdec                   & structure \cr
        &       & \localstrdec                  & local \cr
        &       & \emptystrdec                  & empty \cr
        &       & \seqstrdec                    & sequential\cr
\noalign{\vspace{6pt}}
\strbind & ::=   & \adhocreplacementl{\thenostrsharing}{3cm}{\strbindera}{\barestrbindera} \cr
\noalign{\vspace{6pt}}
\sigexp & ::=   & \encsigexp            & \adhocreplacementl{\thenostrsharing}{-4cm}{generative}{basic}\cr
        &       & \sigid                & signature identifier\cr
\adhocinsertion{\thenostrsharing}{0mm}{&&\rlap{$\sigexp$}\hskip13mm \boxml{where type} & type realisation\cr
        &       & \hskip13mm \boxml{$\tyvarseq$ $\longtycon$ =  $\ty$}\cr}
\noalign{\vspace{6pt}}
\sigdec & ::=   & \singsigdec           & \adhocdeletion{\thenostrsharing}{0mm}{single}\cr\noalign{\vspace{6pt}}\adhocdeletion{\thenostrsharing}{0mm}{        &       & \emptysigdec          & empty\cr
        &       & \seqsigdec            & sequential\cr}%\noalign{\vspace{6pt}}
\sigbind & ::=   & \sigbinder \cr
\noalign{\vspace{6pt}}
}
\makeatother
\vspace{6pt}
\caption{Grammar: Structure and Signature Expressions\index{16.1}}
\label{str-syn}
\end{figure}

\subsection{Syntactic Restrictions}
\label{synresmod.sec}
\begin{itemize}
\item No\index{16.2} binding $\strbind$, $\sigbind$, or $\funbind$ may bind the
      same identifier twice.
\item No description $\valdesc$, $\typdesc$, $\datdesc$, 
      $\exndesc$
      \replacement{\thenofuncspec}{, $\strdesc$ or $\fundesc$}{or $\strdesc$}
      may describe the same identifier
      twice; this applies also to value \replacement{\theidstatus}{constructors}{identifiers} within a $\datdesc$.
\insertion{\thecommentary}{
\item No ${\it tyvarseq}$ may contain the same ${\it tyvar}$ twice.}
\insertion{\thenostrsharing}{
\item 
      Any $\tyvar$ occurring on the right side of a $\datdesc$ of the form
 $\tyvarseq \;\tycon\;\boxml{=}$ $\cdots$ must occur
      in the $\tyvarseq$; similarly, in signature expressions of the form\hfill\break $\sigexp\ \boxml{where type}\, \tyvarseq\,\longtycon\,$
$\boxml{=}\,\ty$, any $\tyvar$ occurring in $\ty$ must occur in $\tyvarseq$.}
\insertion{\thecommentary}{\item No $\datdesc$, $\valdesc$ 
or $\exndesc$ may describe 
$\TRUE$, $\FALSE$, $\NIL$, \boxml{::} or $\REF$.
No $\datdesc$ or $\exndesc$ may describe {\tt it}.
}
\end{itemize}
%\clearpage %containing figure 'Grammar: Specifications'
\begin{figure}[m]
\vspace{4pt}
\makeatletter{}
\tabskip\@centering
\halign to\textwidth
{#\hfil\tabskip1em&\hfil$#$\hfil&#\hfil&#\hfil\tabskip\@centering\cr
\spec	& ::=	& \valspec		& value\cr
	&	& \typespec		& type\cr
	&	& \eqtypespec		& eqtype\cr 
	&	& \datatypespec		& datatype\cr
\adhocinsertion{\thedatatyperepl}{2cm}{&	& \datatypereplspec		& replication\cr}
	&       & \exceptionspec        & exception\cr
        &	& \structurespec	& structure\cr\adhocdeletion{\thetypabbr}{15mm}{&	& \sharingspec	        & sharing
	&	& \localspec    	& local\cr
        &       & \openspec             & open ($n\geq 1$)\cr }\adhocreplacementtexl{\thesingleincludespec}{0mm}{        &       & \inclspec           
  & include ($n\geq 1$)\cr}{        &       & \singleinclspec             & include \cr} 
        &       & \emptyspec            & empty \cr
        &       & \seqspec              & sequential\cr
\adhocinsertion{\thetypabbr}{0mm}{&     & \boxml{$\spec$ sharing type}      &  sharing\cr &&\boxml{\qquad$\longtycon_1$ = $\cdots$ = $\longtycon_n$}& $(n\geq 2)$\cr}
\noalign{\vspace{6pt}}
\valdesc & ::=   & \adhocreplacementl{\theidstatus}{3cm}{\valdescription}{\valviddescription} \cr
\noalign{\vspace{6pt}}
\typdesc & ::=   & \typdescription \cr
\noalign{\vspace{6pt}}
\datdesc & ::=   & \datdescription \cr
\noalign{\vspace{6pt}}
\condesc & ::=   & \adhocreplacementl{\theidstatus}{3cm}{\condescription}{\conviddescription} \cr
\noalign{\vspace{6pt}}
\exndesc & ::=   & \adhocreplacementl{\theidstatus}{3cm}{\exndescription}{\exnviddescription} \cr
\noalign{\vspace{6pt}}
\strdesc & ::=   & \strdescription \cr
\noalign{\vspace{6pt}}
\adhocdeletion{\thenostrsharing}{0mm}{\shareq & ::=   & \strshareq            & structure sharing\cr
        &       &                       & \qquad ($n\geq 2$) \cr
        &       & \typshareq            & type sharing \cr
        &       &                       & \qquad ($n\geq 2$) \cr
        &       & \multshareq           & multiple\cr\noalign{\vspace{6pt}}
\noalign{\vspace{6pt}}}
\cr}
\makeatother
\vspace{3pt}
\caption{Grammar: Specifications\index{17}}
\label{spec-syn}
\end{figure}
%\clearpage %starting with Figure 'Grammar: Functors and Top-level declarations'

\note{\thenofuncspec}{Figure 8: Functor signature expressions and
functor specifications have been removed. Sequential and empty
functor declarations and signature declarations have been removed.
The grammar for $\topdec$ now allows sequencing (without semicolon) 
of structure-level declarations, signature declarations and functor
declarations.}
%
%   1990 Definition: 
%
%\begin{figure}[h]
%\vspace{4pt}
%\makeatletter{}
%\tabskip\@centering
%\halign to\textwidth
%{#\hfil\tabskip1em&\hfil$#$\hfil&#\hfil&#\hfil\tabskip\@centering\cr
%\fundec & ::=  & \singfundec           & single\cr
%        &      & \emptyfundec          & empty\cr
%        &      & \seqfundec            & sequence\cr\cr
%\noalign{\vspace{6pt}}
%\funbind & ::= & \funstrbinder        & functor binding \cr
%         &     & \qquad\qquad\qquad\qquad\optfunbinda \cr}
%\vspace*{6pt}
%\halign to\textwidth
%{#\hfil\tabskip1em&\hfil$#$\hfil&#\hfil&#\hfil\tabskip\@centering\cr
%\funsigexp & ::= & \longfunsigexpa       & functor signature expression\cr
%\noalign{\vspace{6pt}}
%\funspec & ::= & \singfunspec          & functor specification\cr
%        &     & \emptyfunspec           & empty\cr
%        &     & \seqfunspec             & sequence\cr
%\noalign{\vspace{6pt}}
%\fundesc & ::= & \longfundesc\cr
%\noalign{\vspace{6pt}}
%\topdec  & ::= & \strdec               & structure-level declaration\cr
%         &     & \sigdec               & signature declaration\cr
%         &     & \fundec               & functor declaration\cr}
%\noalign{\vspace{12pt}\hskip1cm\parbox{14cm}{{\sl Note:}\/  No $\topdec$ may
%contain, as an initial segment, a shorter top-level declaration followed by a semicolon.}
%}
%}
%\makeatother
%\vspace{6pt}
%\caption{Grammar: Functors and Top-level Declarations\index{18.1}}
%\label{prog-syn}
%\end{figure}


\begin{figure}[h]
\vspace{4pt}
\makeatletter{}
\tabskip\@centering
\halign to\textwidth
{#\hfil\tabskip1em&\hfil$#$\hfil&#\hfil&\quad#\hfil\tabskip\@centering\cr
\fundec & ::=  & \singfundec           & \cr
\noalign{\vspace{6pt}}
\funbind & ::= & \barefunstrbinder     & functor binding \cr
         &     & \qquad\qquad\qquad\qquad\optfunbinda \cr
\noalign{\vspace*{6pt}}
\topdec  & ::= & \strdecintopdec               & structure-level declaration\cr
         &     & \sigdecintopdec               & signature declaration\cr
         &     & \fundecintopdec               & functor declaration\cr
\noalign{\vspace{12pt}\hskip1cm\parbox{14cm}{{\sl Restriction:}\/  No $\topdec$ may
contain, as an initial segment, a $\strdec$ followed by a semicolon.}
}
}
\makeatother
\vspace{6pt}
\caption{Grammar: Functors and Top-level Declarations\index{18.1}}
\label{prog-syn}
\end{figure}


\deletion{\thenoclosurerestriction}{
\subsection{Closure Restrictions}
\label{closure-restr-sec}
The\index{18.2} semantics presented in later sections requires no restriction on
reference to non-local identifiers. For example, it allows a signature 
expression to refer to external signature identifiers and
(via ~$\SHARING$~ or ~$\OPEN$~) to external structure identifiers; it also
allows a functor to refer to external identifiers of any kind.

However, implementers who want to provide a simple facility for
separate compilation may want to impose the following restrictions
(ignoring references to identifiers bound in the initial basis
$\B_0$, which may occur anywhere):

%However, in the present version of the language,
%apart from references to identifiers bound in the initial basis $B_0$
%(which may occur anywhere), it is required that signatures only refer
%non-locally to signature identifiers and that functors only
%refer non-locally to functor and signature identifiers.
%These restrictions ease separate
%compilation; however, they may be relaxed in a future version of the language.
%
%More precisely, the restrictions are as follows (ignoring reference to
%identifiers bound in $B_0$):
\begin{enumerate}
\item In any signature binding ~$\sigid\ \mbox{{\tt =}}\ \sigexp$~,
the only non-local
references in $\sigexp$ are to signature identifiers.
\item In any functor description ~$\funid\ \longfunsigexpa$~,
the only non-local
references in $\sigexp$ and $\sigexp'$ are to signature identifiers,
except that $\sigexp'$ may refer to $\strid$ and its components.
\item In any functor binding ~$\funstrbinder$~, the only non-local
references in $\sigexp$, $\sigexp'$ and $\strexp$ are to functor and signature
identifiers,
except that both $\sigexp'$ and $\strexp$ may refer to $\strid$ and
its components.
\end{enumerate}
In the last two cases the final qualification allows, for example, sharing
constraints to be specified between functor argument and result.
(For a completely precise definition of these closure restrictions,
see the comments to rules \ref{single-sigdec-rule} 
(page~\pageref{single-sigdec-rule}), 
\ref{singfunspec-rule} (page~\pageref{singfunspec-rule})
and \ref{singfundec-rule} (page~\pageref{singfundec-rule})
in the static semantics of modules, Section~\ref{statmod-sec}.)

The\index{19.1} 
significance of these restrictions is that they may ease separate
compilation; this may be seen as follows. If one takes a {\sl module}
to be a sequence of signature declarations, functor specifications
and functor declarations satisfying the above restrictions then the
elaboration of a module can be made to depend on the initial
static basis alone (in particular, it will not rely on
structures outside the module). Moreover, the elaboration 
of a module cannot create new free structure or type names, so 
name consistency (as defined in Section~\ref{consistency-sec}, 
page \pageref{consistency-sec}) is automatically preserved
across separately compiled modules. On the other hand,
imposing these restrictions may force the programmer to write
many more sharing equations than is needed if functors
and signature expressions can refer to free structures.
}


